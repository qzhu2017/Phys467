\lecture{28}{Bose-Einstein Condensation}{Qiang Zhu}{scribe-name1,2,3}

\section{The statistical behavior for Bosons}
As stated in the introduction to Fermions and Bosons, quantum statistics starts to play a role in dense system and low temperatures. For an atom at room temperature, the quantum volume is
\begin{equation}
\epsilon_0 = \frac{h^2}{8mL^2}(1^2+1^2+1^2) = \frac{3h^2}{8mL^2}
\end{equation}

\begin{equation}
N_0 = \frac{1}{e^{(\epsilon_0-\mu)/kT}-1}
\end{equation}

When $T$ is very small, $N_0$ will be quite large. In this case, the denominator must be small,

\begin{equation}
N_0 = \frac{1}{1+(\epsilon_0-\mu)/kT-1} = \frac{kT}{\epsilon_0-\mu}  ~~~~(\textrm{when~} N_0 \gg 1)
\end{equation}

The chemical potential $\mu$ must be equal to $\epsilon_0$ at T=0, and just a bit less than $\epsilon_0$ at small $T$. %To calculate the total energy of all electrons, we need to sum over the energies of the electrons in all occupied states. 
Now the question is \textbf{at which temperature we can observe that $N_0$ remains very large?}


\section{Computing the total number of Bosons}
\begin{equation}
N = \sum_s \frac{1}{e^{(\epsilon_s-\mu)/kT}-1}
\end{equation}

In practice, we can turn it to an integral,
\begin{equation}
\label{eq0}
N = \int_0^\infty g(\epsilon)\frac{1}{e^{(\epsilon_s-\mu)/kT}-1}d\epsilon
\end{equation}

Where $g(\epsilon)$ is the density of states, which has a similar form following the electron gas model. 
\begin{equation}
g(\epsilon) = \frac{2}{\sqrt{\pi}}\bigg(\frac{2\pi m}{h^2}\bigg)^{3/2}V\sqrt{\epsilon}
\end{equation}

\begin{figure}[h]
\centering
\includegraphics[width=0.9\linewidth]{imgs/BEC1.png}
\caption{The distribution of bosons as a function of energy is the product of
two functions, the density of states and the Bose-Einstein distribution. Copyright 2000, Addison-Wesley. }
\end{figure}

The trouble is that we cannot evaluate eq.(\ref{eq0}) analytically. In order to work it out, we must guess some value for the $\mu$ term. A good starting point is let $\mu$=0. Changing the variable to $x = \epsilon/kT$
\begin{equation}
\begin{split}
    N = & \frac{2}{\sqrt{\pi}}\bigg(\frac{2\pi m}{h^2}\bigg)^{3/2}V \int_0^{\infty} \frac{\sqrt{\epsilon} d\epsilon}{e^{\epsilon/kT}-1}\\
      = & \frac{2}{\sqrt{\pi}}\bigg(\frac{2\pi mkT}{h^2}\bigg)^{3/2}V \int_{0}^{\infty} \frac{\sqrt{x} dx}{e^x-1}\\
\end{split}
\end{equation}

The integral over $x$ gives 2.315, which leaves us with
\begin{equation}
N = 2.612\bigg(\frac{2\pi mkT}{h^2}\bigg)^{3/2}V
\end{equation}

That result is wrong! It means that the number of particles purely depends on $T$. In fact, there can be only one $T$ corresponds to this value.
\begin{equation}
    N = 2.612\bigg(\frac{2\pi m kT_c}{h^2}\bigg)^{3/2}V
\end{equation}

When $T<T_c$, this will be no longer valid during the transformation from summation to integral. This is because the terms from $\epsilon =0$ are missing. Therefore, it should be better expressed as
\begin{equation}
    N_\textrm{exited} = 2.612\bigg(\frac{2\pi mkT}{h^2}\bigg)^{3/2}V
\end{equation}

While the gap between $N$ and $N_\textrm{exited}$ is the number of atoms in the ground state.
\begin{equation}
    N_0 = N-N_\textrm{exited} = \bigg[1-\bigg(\frac{T}{T_c}\bigg)^{3/2}\bigg]N
\end{equation}

\begin{figure}[ht]
\centering
\includegraphics[width=0.8\linewidth]{imgs/BEC2.png}
\caption{Number of atoms in the ground state ($N_0$) and in excited states,
for an ideal Bose gas in a three-dimensional box. Below $T_c$ the number of atoms in excited states is proportional to $T^{3/2}$. Copyright 2000, Addison-Wesley}
\end{figure}

The abrupt accumulation of atoms in the ground state below $T_c$ is called \textbf{Bose-Einstein condensation}.

\section{Real World Examples}
In 1995 BEC of a gas of weakly interacting atoms was first achieved using Rb-87. Later, BEC was achieved with dilute gases of atomic Li, Na, H, .etc.

The superfluid phase of $^4$He is also considered to be an example of BEC.

\section{Why does it happen?}



