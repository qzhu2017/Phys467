
\lecture{25}{Gibbs Factor, Bosons, Fermions}{Qiang Zhu}{scribe-name1,2,3}


\section{The Gibbs Factor}
Remember when the derived the Boltzmann factor in the previous chapter, we wrote the ratio of probabilities of 
two different states as follows,
\begin{equation} 
\frac{P(s_2)}{P(s_1)} = \frac{\Omega_R(s_2)}{\Omega_R(s_1)} 
                      = \frac{e^{S_R(s_2)/k}}{e^{S_R(s_2)/k}}
                      = e^{[S_R(s_2)-S_R(s_1)]/k}
\end{equation}

According to the thermodynamic identify,
\begin{equation}
dS_R = \frac{1}{T} (dU_R + PdV_R - \mu dN_R) 
\end{equation}

When constructing the Boltzmann factor, we ignored $PdV$ and $\mu dN$ terms. However, we want to keep the  $\mu dN$ to account
for the exchanges in the number of particles. 
\begin{equation}
\frac{P(s_2)}{P(s_1)} = \frac{e^{[-E(s_2)-\mu N(s_2)]/kT}}{e^{[-E(s_1)-\mu N(s_1)]/kT}}
\end{equation}

And here we call the exponential factor Gibbs factor.

Accordingly, the probability functions is written as
\begin{equation}
P(s) = \frac{1}{Z}e^{-E(s)/kT}
\end{equation}

Where $Z$ is called the grand partition function
\begin{equation}
Z = \sum e^{[-E(s)-\mu N(s)]/kT}
\end{equation}

If more than one type of particle can be present, the $\mu dN$ term becomes $\sum {\mu_i dN_i}$

The grand partition is very useful in dealing with the situation where the particles exchange during the process.


\section{Bosons and Fermions}
More importantly, the concept of Gibbs factors is very useful in quantum statistics, where the study of dense systems in which two
or more identifcal particles have reasonable chance to occupy the same single-particle state.

Bosons: phontons, helium-4, integer spin

Fermions: electrons, protons, neutrons, half-integer spin.

When the system is very dense, namely $Z1 \gg N$, this becomes important.
Quantum length/volume.

\section{The Distribution Functions}
\begin{equation}
P(n) = \frac{1}{Z}e^{-(n\epsilon-\mu n)/kT} = \frac{1}{Z} e^{-n(\epsilon-\mu/kT}
\end{equation}

If the particles are fermions, $n$ could be only 0 or 1, so the grand partition function becomes
\begin{equation}
Z = 1 + e^{-(\epsilon-\mu)/kT}
\end{equation}

And the average number of particles is
\begin{equation}
\begin{split}
\bar{n} = \sum_n{nP(n)} &= 0\cdot P(0) + 1\cdot P(1) \\
                        &= \frac{e^{-(\epsilon-\mu)/kT}}{1+e^{-(\epsilon-\mu)/kT}}\\
                        &= \frac{1}{e^{-(\epsilon-\mu)/kT}+1}
\end{split}
\end{equation}


If the particles are Bosons,
\begin{equation}
\begin{split}
Z   & = 1 + e^{-(\epsilon-\mu)/kT} + e^{-2(\epsilon-\mu)/kT} + ...\\
    & = 1 + e^{-(\epsilon-\mu)/kT} + (e^{-(\epsilon-\mu)/kT})^2 + ...\\
    & = \frac{1}{1- e^{-(\epsilon-\mu)/kT}}
\end{split}
\end{equation}

and
\begin{equation}
\bar{n} = \sum_n{nP(n)} = 0\cdot P(0) + 1\cdot P(1) + ....
\end{equation}

Similar to what we did on $\bar{E}$, let $x = (\epsilon-\mu)/kT$,
\begin{equation}
\bar{n} = \sum_n{n \frac{e^{-nx}}{Z}} = -\frac{1}{Z} \sum{\frac{\partial e^{-nx}}{\partial x}} = -\frac{1}{Z}\frac{\partial Z}{\partial x}
\end{equation}

Therefore,
\begin{equation}
\begin{split}
\bar{n} & = -\frac{1}{1-e^{-x}}\frac{\partial (1-e^{-x})^{-1}}{\partial x} = \\
        & = \frac{1}{e^{(\epsilon-\mu)/kT}-1}
\end{split}
\end{equation}

For the classical particles,
\begin{equation}
P(s) = \frac{1}{Z} e^{-\epsilon/kT}
\end{equation}
and
\begin{equation}
\bar{n} = NP(s) = \frac{N}{Z1}e^{\epsilon/kT}
\end{equation}
According to the result of Problem 6.44, the chemical potential for such system is
\begin{equation}
\mu = -kT\textrm{ln}(Z_1/N)
\end{equation}
\begin{equation}
\bar{n} = NP(s) = \frac{N}{Z1}e^{\epsilon/kT}=e^{\mu/kT}e^{\epsilon/kT} = e^{-(\epsilon-\mu)/kT}
\end{equation}


Show the plots of $n$ versus $\epsilon$ for Bosons, Fermions and Boltzmann particles.
Clearly, the three distributions become equal when $(\epsilon-\mu)/kT \gg 1$.
For any of these applications, we can apply the distributions as long as we know the chemical potential.


