\lecture{7}{The Second Law and Entropy}{Qiang Zhu}{scribe-name1,2,3}


\section{Calculate $\Omega$ for a Einstein Solid}
Remember we have done some computer programmings for two-state systems.
A general trend is that it approaches to a large number $N$, $\Omega(N)$ tends to be localized.
If we treat the histogram of $\Omega(N)$ as a continues function (true when $N$ is very large),
it looks like a very smooth curve and $\Omega(N)$ follows some kind of distribution. 

Now let's try to figure out what it is, by taking a Einstein solid as an example. 
\begin{equation} \label{E0}
 \Omega(N, q) = \binom{q+N-1}{q} = \frac{(q+N-1)!}{q!(N-1)!}
\end{equation}

In reality, there are always many more energy units ($q$) than oscillator ($N$), so we assume $q\gg N$.

To make it easier, let's just remove \textit{-1} in eq. \ref{E0},
\begin{equation} \label{E1}
\begin{split}
 \text{ln}\Omega &= \text{ln}(\frac{(q+N)!}{q!N!})\\
         &= \text{ln}(q+N)! - \text{ln}q! - \text{ln}N!\\
   &\approx (q+N)\text{ln}(q+N) - (q+N) - q\text{ln}q - q - N\text{ln}N - N\\
        & = (q+N)\text{ln}(q+N) - q\text{ln}q - N\text{ln}N \\
\end{split}
\end{equation}

Remember we have the assumption of $q\gg N$, namely ($q/N \rightarrow$ 0)
\begin{equation} \label{E2}
       \text{ln}(q+N) = \text{ln}(q(1+\frac{N}{q}))\\
                      = \text{ln}q + \frac{N}{q} 
\end{equation}

Therefore, we have 
\begin{equation} \label{E3}
       \text{ln}\Omega = N\text{ln}\frac{q}{N} + N + \frac{N^2}{q}\\
                 \approx N\text{ln}\frac{q}{N} + N 
\end{equation}

If we remove the logarithm sign,
\begin{equation} \label{E4}
       \Omega(N,q) = (\frac{eq}{N})^N
\end{equation}


\section{Calculate $\Omega$ for two Einstein Solids}
Naturally, we now know the general form of two Einstein Solids model,
\begin{equation} \label{E5}
       \Omega(N_A,q_A,N_B,q_B) = (\frac{eq_A}{N_A})^{N_A} (\frac{eq_B}{N_B})^{N_B}
\end{equation}

For simplicity, let make $N_A = N_B = N $, then
\begin{equation} \label{E6}
       \Omega(N,q_A,q_B) = (\frac{e}{N})^{2N} (q_A q_B)^{N}
\end{equation}

Based on what we plot in the homework, we know that $\Omega$ reaches its maximum value at 
$q_A = q_B = q$/2,
\begin{equation}
       \Omega_\text{max} = (\frac{e}{N})^{2N} (q/2)^{2N}
\end{equation}

Now let's try to calculate the points near $q$/2, say,
\begin{equation}
  q_A = q/2+x, ~~~~~~~~~ q_B=q/2-x.
\end{equation}

Using eq. \ref{E6},
\begin{equation} \label{E7}
       \Omega(N,q,x) = (\frac{e}{N})^{2N} ((\frac{q}{2})^2-x^2)^{N}.
\end{equation}

To simplify it, we get
\begin{equation} \label{E8}
\begin{split}
       \text{ln}((\frac{q}{2})^2-x^2)^{N} 
           &= N[\text{ln}[(\frac{q}{2})^2-x^2]\\
           &= N[\text{ln}(\frac{q}{2})^2 + \text{ln}(1-(\frac{2x}{q})^2))]\\
           &= N[\text{ln}(\frac{q}{2})^2 - (\frac{2x}{q})^2)]\\
           &= N[\text{ln}(\frac{q}{2})^2 - (\frac{2x}{q})^2)]
\end{split}                                          
\end{equation}
hence we have 
\begin{equation} \label{E9}
  \Omega = \Omega_\text{max} \cdot e^{-N(2x/q)^2}
\end{equation}
This is a typical Gaussian function. A standard version is as follows,
\begin{equation}
  f(x|\mu,\delta^2) = \frac{1}{\sqrt{2\delta^2}\pi} e^\frac{-(x-\mu)^2}{2\delta^2}.
\end{equation}

\begin{tikzpicture}[x=1.5cm, y=1.5cm]

%normal distribution function 'normaltwo'
    \def\normaltwo{\x,{4*1/exp(((\x-3)^2)/2)}}
% input y parameter
    \def\y{3.5}
% this line calculates f(y)
    \def\fy{4*1/exp(((\y-3)^2)/2)}
% Shade orange area underneath curve.
    %\fill [fill=orange!60] (2.6,0) -- plot[domain=0:4.4] (\normaltwo) -- ({\y},0) -- cycle;
% Draw and label normal distribution function
    \draw[color=blue,domain=0:6] plot (\normaltwo) node[right] {};
% Add dashed line dropping down from normal.
    \draw[dashed] ({\y},{\fy}) -- ({\y},0)     node[below] {$\mu+\delta$};
    \draw[dashed] ({6-\y},{\fy}) -- ({6-\y},0) node[below] {$\mu-\delta$};
% Optional: Add axis labels
    \draw (-.2,2.5) node[left] {$f(x)$};
    \draw (3,-.5) node[below] {$x$};
% Optional: Add axes
    \draw[->] (0,0) -- (6.2,0) node[right] {};
    \draw[->] (0,0) -- (0,5) node[above] {};
\end{tikzpicture}
\begin{enumerate}
\item symmetric
%\item unimodal
%\item its second derivetive is equal to 
\item Gaussian width
\end{enumerate}



The multiplicity falls off to 1/e of its maximum when 
\begin{equation}
  N(\frac{2x}{q})^2 = 1 ~~~~~~or~~~~~~ x = \frac{q}{2\sqrt{N}}
\end{equation}

Let's plug in some number, say $N$=\textrm {$10^{20}$}, 
This results tell us,
when two Einstein solids are in thermodynamical equilibrium, any random fluctuation
will be not measurable. The most-likely macrostates are very localized.\\
{\bf Exercises}
\begin{enumerate}
\item Problem 2.20. 
\item Problem 2.23
\end{enumerate}

\section{Ideal Gas}
Suppose we have a single gas atom (Ar), with kinetic energy $U$ in a container of volume $V$, what is its corresponding $\Omega$?
Obviously, the possible microstate is proportional to V. In principle, the atom can stay at any place of $V$.
Also, each microstate can be represented as a vector, since it has velocity (more precisely Momentum). 
Therefore
\begin{equation}
\Omega \approx V\cdot V_p
\end{equation}



It appears that both $V$ and $V_p$ somehow relate to very large numbers, would their product go to infinity? 
Fortunately, we have the famous {\bf Heisenberg uncertainty principle}:
\begin{equation} \Delta{x}\Delta{p_x} = h. \end{equation}

For a one-dimensional chain, we define $L$ as the length in real space, $L_p$ as the length in momentum space,
\begin{equation} \Omega_{1D} = \frac{L}{\Delta{x}} \frac{L_p}{\Delta{p_x}} = \frac{LL_p}{h}. \end{equation}
Therefore, its 3D version is,
\begin{equation} \Omega_1 = \frac{VV_p}{h^3}. \end{equation}
Accordingly, the multiplicity function for an ideal gas of two molecules should be
\begin{equation} 
\Omega_2 = \frac{1}{2}\frac{V^2}{h^6} \times \text{area of $P$ hypersphere}
\end{equation}
if the two molecules are indistinguishable.
The general form for $N$ should be 
\begin{equation} \Omega_N = \frac{1}{N!}\frac{V^N}{h^{3N}} \times \text{area of $P$ hypersphere}. \end{equation}

For $N$=1, how to calculate the area?
Since $U$ depends on the momentum by
\begin{equation}
U = \frac{1}{2}m(v_x^2+v_y^2+v_z^2)=\frac{1}{2m}(p_x^2+p_y^2+p_z^2)
\end{equation}
\begin{equation}
p_x^2+p_y^2+p_z^2 = \sqrt{2mU}
\end{equation}

So the momentum space is the surface of a sphere with radius $\sqrt{2mU}$, namely,
\begin{equation}
\begin{split}
\text{area} = & 2    ~~~~~~~~~~~~~~~~~~~~~~~~\text {($d$=1)}\\
            = & 2\pi r    ~~~~~~~~~~~~~~~~~~~\text {($d$=2)}\\
            = & 4\pi r^2    ~~~~~~~~~~~~~~~~~\text {($d$=3)}\\
            = & ....\\
            = & ....\\
            = & \frac{2\pi^{d/2}}{\Gamma(d/2)} r^{d-1} ~~~~~\text {($d$ in general)}\\
\end{split}
\end{equation}

Therefore, the general $\Omega$ is
\begin{equation} 
\Omega_N = \frac{1}{N!} \frac{V^N}{h^{3N}} \frac{2\pi^{3N/2}}{(3N/2-1)!} \sqrt{2mU}^{3N-1}. 
\end{equation}


\begin{equation}
\Omega(U,V,N) = f(N)V^NU^{3N/2}
\end{equation}
where $f(N)$ is a complicated function of $N$.\\

For two interaction gases,
\begin{equation}
\Omega(U,V,N) = [f(N)]^2(V_AV_B)^N(U_AU_B)^{3N/2}
\end{equation}


%\appendix
\section{Appendix: Area of high-dimensional Hypersphere}
For a d-dimensional hypersphere with a radius of $r$, we can solve it iteratively.
When $d$=1, $A(r)$=2, $d$=2, $A(r)=2\pi r$
\begin{equation}
    A_3(r) = \int_0^{\pi} A_2(r\sin\theta)rd\theta = 2\pi r^2 \int_0^{\theta}d\theta = 4\pi r^2.
\end{equation}
Consequently, we can keep doing this
\begin{equation}
\begin{split}
    A_d(r) = & \int_0^\pi{A_{d-1}(r\sin(\theta))rd\theta} \\
           = & \int_0^\pi\frac{2\pi^{(d-1)/2}}{\Gamma(\frac{d-1}{2})}(r\sin\theta)^{d-2}rd\theta \\
           = & \frac{2\pi^{(d-1)/2}}{\Gamma(\frac{d-1}{2})}r^{d-1} \int_0^{\pi}(\sin\theta)^{d-2}d\theta\\
\end{split}
\end{equation}

\begin{equation}
    \int_0^{\pi} (\sin\theta)^nd\theta = \frac{\sqrt{\pi}\Gamma(\frac{n+2}{2})}{\Gamma(\frac{n+2}{2})}
\end{equation}

so
\begin{equation}
    A_d(r) = \frac{2\pi^{d/2}}{\Gamma(d/2)}r^{d-1}
\end{equation}

