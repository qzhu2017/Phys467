% This syllabus template was created by:
% Qiang Zhu
% Document settings
%\documentclass[11pt]{article}
%\usepackage[margin=1in]{geometry}
%\usepackage[pdftex]{graphicx}
%\usepackage{multirow}
%\usepackage{setspace}
%\usepackage{hyperref}
%\usepackage{breakurl}
%\pagestyle{plain}
%\setlength\parindent{0pt}
%\def\UrlBigBreaks{\do\/\do-\do:}
%\begin{document}
% Course information
\begin{tabular}{ l }
  \LARGE Physics 467/667 (Spring 2019)\\
  \LARGE Thermodynamics/Statistical Physics \\
  \LARGE Tues/Thurs 11:30AM - 12:45PM BPB 249 \\
\end{tabular}\\\\
%\vspace{10mm}

% Professor information
\begin{tabular}{ l l }
  \large Instructor &\large Prof. Qiang Zhu \\
  \large Email      &\large qiang.zhu@unlv.edu \\
  %\large Website    &\large \url{http://www.physics.unlv.edu/~qzhu/} \\
  \large Office     &\large BPB 232 \\
  \large Office hrs &\large Tue/Thurs 10:00AM - 11:30AM \\
\end{tabular}\\

%\vspace{5mm}
% Course details
\textbf {Course Outline:}
\begin{itemize} 
\item Fundamentals of thermodynamics, equations of state, laws of thermodynamics, entropy
\item Heat engines and refrigerators
\item Free energy and classical thermodynamics
\item Boltzmann statistics
\item Quantum statistics of ideal gas and simple solid
\end{itemize}

\textbf {Prerequisite:} PHY 182\\
%\textbf {Note(s):} A minimum grade of C is required in this course to progress to COURSE.\\
\textbf {Credit Hours:} 3 \\
\textbf {Textbook:} \emph{An Introduction to Thermal Physics}, D. Schroeder\\

\textbf {Grade Distribution:} \\
%\hspace*{40mm}
\begin{tabular}{ l l }
Assignments        & 20\% \\
Midterm Exams 1/2  & 40\% \\
Final Exam         & 40\% \\
Extra Credits      & 10\% \\
\end{tabular} 

% Course Policies. These are just examples, modify to your liking.
% College Policies
\textbf {\large Course Description}:\\
This course combines elements of classical thermodynamics and statistical physics and covers materials from chapters 1 through 7 in the text book. Approximately, we spend two weeks for each chapter. The weekly coverage might change as it depends on the progress of the class. There will be two exams during the semester. These exams may include both take-home and in-class work. This final exam will cover all materials taught in this semester. You may work with others on the homework, but take-home exams must be done {\bf strictly by yourself}. Barring documentable emergencies or observance of a certifiable religious holiday, all exams must be taken at the time and place specified.
\\
\textbf {Learning Outcomes}:
\begin{itemize} 
\item understand the fundamental principles for thermal physics
\item know how to apply these principles to various applications
\item understand the statistical basis for thermodynamics
\item understand the difference between classical and quantum statistics
\end{itemize}

Please see the Student Syllabus Policies Handout for select, useful information for students. This document can be found at:\\ \url{https://www.unlv.edu/sites/default/files/page_files/27/SyllabiContent-MinimumCriteria-2018-2019.pdf}
%\end{document}

